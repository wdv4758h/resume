%-------------------------------------------------------------------------------
%	SUBSECTION TITLE
%-------------------------------------------------------------------------------
\cvsubsection{Open Source Contribution}

%-------------------------------------------------------------------------------
%	CONTENT
%-------------------------------------------------------------------------------
\begin{cvopensources}

% Project Name
% Project Description
% Link
% Contribution Detail

%---------------------------------------------------------
  \cvopensource
    {Servo}
    {The Servo Browser Engine}
    {https://github.com/servo/servo/commits/master?author=wdv4758h}
    {
      \begin{cvitems}
        % 這是要在 JavaScript 端使用的 FormData 實做
        % 原先沒有實做 Iterable,所以在 WebIDL 中註解掉了
        % 這 PR 實做了相關功能
        % 其中注意到目前的資料是用 HashMap 儲存,而 Gecko 中是用 Array
        % 而且 FormData 的規範說要保留順序
        % 所以這是未來需要修改部份
        % 未解 Issue: https://github.com/servo/servo/issues/13105
        \item {Added FormData Iterable}
        % 更改 Referer Policy 為 Origin,"foo.com/a.html" 會被轉成 "foo.com"
        % CORS 是指和原本 Domain 不同的 Request,例如 Domain A 的網頁中有個圖片是連到 Domain B
        % Preflighted Request 是指先向 Domain B 發送 OPTIONS HTTP method
        % 藉此來判斷是否可以安全地發送真正的 request
        % https://developer.mozilla.org/en-US/docs/Web/HTTP/Access_control_CORS#Preflighted_requests
        % https://fetch.spec.whatwg.org/#cors-preflight-fetch
        \item {Solved wrong referrer policy in cors\_preflight\_fetch}
        % 原本的 pipeline_id 因為其他 issue 而暫時用假造的,改成從 NetworkEvent 中取出真正的 pipeline_id
        \item {Added using real pipeline ID value}
        % 有段程式碼可以直接拿 Ownership,但是原程式碼還用 Borrow 後再 Clone 以取得可以回傳的資料
        \item {Remove unnecessary clone from ServiceWorkerManager::prepare\_activation}
      \end{cvitems}
    }

%---------------------------------------------------------
  \cvopensource
    {rust-rocksdb}
    {RocksDB wrapper for Rust}
    {https://github.com/ethcore/rust-rocksdb/commits/master?author=wdv4758h}
    {
      \begin{cvitems}
        % 官方 RocksDB 其實有 FreeBSD 支援,但是這 Binding 的 Build Script 沒有用對應的參數打開他
        \item {Fix FreeBSD build}
      \end{cvitems}
    }

%---------------------------------------------------------
  \cvopensource
    {Parity}
    {Fast, light, robust Ethereum implementation}
    {https://github.com/ethcore/parity/commits/master?author=wdv4758h}
    {
      \begin{cvitems}
        % 開啟 FreeBSD 上使用 "open" 指令來用預設瀏覽器開啟 URL
        \item {Fix open function on FreeBSD}
      \end{cvitems}
    }

%---------------------------------------------------------
  \cvopensource
    {nanomsg.rs}
    {Nanomsg wrapper for Rust}
    {https://github.com/ethcore/nanomsg.rs/commits/master?author=wdv4758h}
    {
      \begin{cvitems}
        % 原本的 Binding 會在選擇 Static Linking 時,仍然用到 Dynamic Linking
        % 加上需要的 Attribute 來「真的」 Static Linking
        \item {Don't dynamic link to libnanomsg when using 'bundled' feature}
      \end{cvitems}
    }

%---------------------------------------------------------
  \cvopensource
    {rust}
    {The de-facto Rust compiler implementation}
    {https://github.com/rust-lang/rust/commits/master?author=wdv4758h}
    {
      \begin{cvitems}
        % rustc::plugin 已經被改成 rustc_plugin,但是有個文件說明沒更新到
        \item {Change 'rustc::plugin' to 'rustc\_plugin' in doc comment}
        % 更改 Rust 編譯器錯誤的錯誤訊息
        \item {Fix label messages for E0133}
        % 更新 Rust 編譯器的錯誤訊息,E0138 是定義了多個 "start" 函式,尤其是用 #[start]
        \item {Update E0138 to new format}
        % 更新 Rust 編譯器的錯誤訊息,E0133 是該用 unsafe 但沒用
        \item {Update E0133 to new format}
      \end{cvitems}
    }

%---------------------------------------------------------
  \cvopensource
    {clap-rs}
    {A full featured, fast Command Line Argument Parser for Rust}
    {https://github.com/kbknapp/clap-rs/commits/master?author=wdv4758h}
    {
      \begin{cvitems}   % Contribution Detail
        % 修正巢狀指令時的 Fish Shell 補完支援
        \item {fix(Completions): fish completions for nested subcommands}
        % 產生出 Fish Shell 的補完支援
        \item {feat(Completions): one can generate a basic fish completions script at compile time}
      \end{cvitems}
    }

%---------------------------------------------------------
  \cvopensource
    {CPython}
    {The de-facto reference Python implementation}
    {}
    {
      \begin{cvitems}
        % trafile 內自動透過副檔名偵測需要的演算法出了差錯,修正後補上測試
        \item {Fix choosing of compression algorithm in tarfile CLI}
        % 文件說明有部份跟實際狀況不同,修正說明
        \item {Fix docstring in http.server.test}
      \end{cvitems}
    }

%---------------------------------------------------------
  \cvopensource
    {re2}
    {A fast, safe, thread-friendly alternative to backtracking regular expression engines}
    {https://github.com/google/re2/commits/master?author=wdv4758h}
    {
      \begin{cvitems}
        % 爬 Benchmark 的結果來生 gnuplot 指令,以畫出相關的圖表
        \item {Add benchmark's gnuplot support}
      \end{cvitems}
    }

%---------------------------------------------------------
  \cvopensource
    {Neovim}
    {Vim-fork focused on extensibility and agility}
    {https://github.com/neovim/neovim/commits/master?author=wdv4758h}
    {
      \begin{cvitems}
        % Clang 3.7 換了參數,在 CMake 中偵測後使用最新的參數
        \item {build: fix '-fno-sanitize-recover' warning in Clang 3.7 }
        % 標示一些 Vim 的 Patch 是 Neovim 不需要的
        \item {version.c: mark patches NA}
        % 將 Vim 上的 Patch 轉移到 Neovim
        \item {Porting Vim Patch 7.4.799}
      \end{cvitems}
    }

%---------------------------------------------------------
\end{cvopensources}
